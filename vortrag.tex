\documentclass{beamer} 

% Michael Maier, 2012.
% CC-BY-SA

\usepackage[utf8]{inputenc}
\usepackage[ngerman]{babel}
\usepackage{graphicx}

\title{Mapnik-Toolchain für den eigenen OpenStreetMap-Kartenserver} 
\author{Michael Maier \textless Michael.Maier@student.tugraz.at\textgreater} 
\date{May 12th, 2012} 

\usetheme{Antibes}



%\usebackgroundtemplatei{
%\includegraphics[width=\paperwidth,
%height=0.8\paperheight]{mag_map.png}
%}

\begin{document}

%\maketitle

\begin{frame} 


\begin{figure}
  \centering
  \includegraphics[width=.5\textwidth]{mag_map.png}
\end{figure}

\begin{center}
\Huge{Der OSM-Renderingserver}
\end{center}

\begin{center}
\Large{\emph{Die Toolchain zu eigenen Karten}}
\end{center}

\end{frame}



\begin{frame}{whoami}

  \begin{itemize}
    \item Michael Maier \textless \href{mailto:Michael.Maier@student.tugraz.at}{Michael.Maier@student.tugraz.at}\textgreater
    \item Telematik-Student an der TU Graz seit 2003
    \item Linux-User (Debian/grml) seit 2004
    \item OpenStreetMap seit Juli 2010
    \begin{itemize}
      \item OSM-Username: \emph{species}
      \item Mapping-Area Graz, Leoben
      \item Mit dem Fahrrad, Motorrad und Öffis
    \end{itemize}
  \end{itemize}
\end{frame}

\section{Einleitung}

\begin{frame}{Was ist OpenStreetMap}

\begin{itemize}
  \item OpenStreetMap ist eine freie Weltkarte nach dem Wiki-Prinzip
  \item Entsteht aus der Arbeit von weltweit 600.000 Hobbykartografen
  \item Das komplette ``planet file'' ist inzwischen ca. 250\,GB groß (xml)
\end{itemize}


\end{frame}

\begin{frame}{Wer steckt dahinter}
  \begin{itemize}
    \item Menschen wie Du und ich ... "\emph{Mapper}"
    \item Die OpenStreetMap Foundation
    \item Organisationen, die die Verwendung ihrer Daten erlauben \\
    zB Luftbilder von:
    \begin{itemize}
      \item Yahoo
      \item Bing
      \item Geoimage.at
    \end{itemize}
  \end{itemize}

\end{frame}



\section{Warum OpenStreetMap}

\begin{frame}{Warum OpenStreetMap?}

Nachteile kommerzieller Anbieter:

\begin{itemize}
  \item Restriktive Lizenzen - only Free as in Beer
  \item Offline-Nutzung oft nicht erlaubt - Roaming!
  \item Absichtliche Fehler
  \item Änderungen/Richtigstellungen?
  \pause
  \item Bsp Google TOS: Durch die Nutzung schließen sie einen rechtsgültigen Vertrag mit Google - Dürfen unmündige Personen (unter 18?) Google Maps überhaupt nutzen?
  \item Kosten! Google verlangt ab 25K API-Zugriffen/Tag!
\end{itemize}

\end{frame}

\begin{frame}{Warum OpenStreetMap?}

Vorteile von OpenStreetMap:

\begin{itemize}
  \item Rohdaten sind frei verfügbar
  \item Jeder kann Dinge ändern
  \item Freie Karten für Navis
  \item Karten mit eigenem Stil rendern
\end{itemize}

Freiheit schafft Möglichkeiten!

\end{frame}


\section{Wie funktioniert OpenStreetMap}

\begin{frame}{Wie funktioniert OpenStreetMap?}

\begin{itemize}
  \item Zentrale Datenbank
  \item Jeder kann Daten hinzufügen/ändern
  \item Qualitätskontrolle durch jeden!
  \item Jeder kann Daten/Extrakte runterladen
\end{itemize}

\end{frame}


\subsection{Daten}
\begin{frame}{Datenformat}

\begin{itemize}
  \item Punkte (Koordinaten), $\Rightarrow$ ``Node'' \includegraphics[width=0.5cm]{node.png}
  \item Flächen sind eine Reihe von Nodes, $\Rightarrow$ ``Way'' \includegraphics[width=0.5cm]{way.png}
  \item Gruppierungen von Ways $\Rightarrow$ ``Relations'' \includegraphics[width=0.5cm]{relation.png}
\end{itemize}

\pause

Jedes Element hat Eigenschaften $\Rightarrow$ ``Tags'', zB:
\begin{itemize}
  \item amenity = Cafe \includegraphics[width=0.5cm]{cafe.png}
  \item highway = footway \includegraphics[width=1cm]{footway.png}
  \item building = yes  \includegraphics[width=0.5cm]{building.png}
  \item landuse = farmland 
\end{itemize}

\pause

Was taggen wir?

\pause

Alles :-)

\begin{itemize}
  \item highway=*, landuse=*, shop=*, tourism=*, \dots
  \item ?=*
\end{itemize}

Gebräuchliche Tags und Beschreibungen $\Rightarrow$ Wiki!

\end{frame}

\section{Kartenserver}

\begin{frame}{Quellen?}
  Alles aus dem Wiki:

  \begin{itemize}
              \item  Howto: Minutely Mapnik             \url{http://wiki.openstreetmap.org/wiki/DE:Minutely\_Mapnik}
  \item   \url{http://wiki.osm.org/Mapnik}
  \item \url{http://wiki.osm.org/Tirex}
  \item \url{http://wiki.openstreetmap.org/wiki/Contours\_on\_the\_Cycle\_Map}
\end{itemize}


\end{frame}



\begin{frame}{Übersicht}
  Wie kommt man zur Slippy Map?
  \pause
  \begin{itemize}
        \item Javascript-API im Browser
        \item Tiles werden vom Renderer erzeugt
        \item Renderjobs werden von Queue-Manager verwaltet
        \item Queue-Manager bekommt Tasks vom Webserver-Modul
        \item Webserver liefert Tiles aus
        \item Hintergrund: 
          \begin{itemize}
                  \item GIS-Datenbank Postgres/PostGIS
                  \item DB-Fill mit Osmosis/osm2pgsql
                  \item Minutely Updates
         \end{itemize}
       \end{itemize}


\end{frame}


\subsection{Grundlagen}

\begin{frame}{Userseitig}
Wie funktioniert die Kartenanzeige im Browser?
\pause

\begin{itemize}
  \item Javascript-Framework lädt on Demand Kacheln (Tiles) vom Server
\end{itemize}
zB
\begin{itemize}
  \item \href{http://www.openlayers.org/}{OpenLayers}
  \item \href{http://leaflet.cloudmade.com/}{Leaflet} by Cloudmade
\end{itemize}

\end{frame}

\begin{frame}{Tiles?}
  \begin{itemize}
        \item Ein Tile ist ein 256x256px PNG.

        \item Tiles werden zB folgendermaßen referenziert:
  \url{http://tile.openstreetmap.org/7/63/42.png}
  
  \$Hostname/\$tiledirs/\$zoom/\$x/\$y.png

	\item Serverseitig werden sog. \emph{Metatiles} verwendet, die aus 8x8 Tiles bestehen. (Weniger Renderaufwand)
  \end{itemize}

\end{frame}

\subsection{Serverseitig}
\begin{frame}{Renderer:Mapnik}
  Portal siehe \url{http://wiki.osm.org/Mapnik}
  \begin{itemize}
              \item Python
              \item Stile in XML
              \item Daten von Postgres oder Files
            \end{itemize}

\end{frame}

\begin{frame}{Queue-Manager:Tirex}
  Sprich: T-Rex
  siehe \url{http://wiki.osm.org/Tirex}
  \begin{itemize}
                    \item Multi-CPU unterstützung
                    \item Job-Verteilung auf mehrere Rechner
                    \item Steuerung durch:
                      \begin{itemize}
                      \item wird vom Apache \emph{mod-tile} gesteuert
                      \item batch-jobs per skript
                    \end{itemize}
                  \item Munin/Nagios-Überwachung
                \end{itemize}


\end{frame}

\begin{frame}{Webserver: Apache mit \emph{mod-tile}}
  Liefert aus gerenderten Metatiles ausgeschnittene Tiles aus.
  \pause
  Sehr simples Modul...
  \begin{itemize}
                          \item Zoomlevel hardcoded $\Rightarrow$ recompile for z19
                          \item metatiles muessen in  /var/lib/mod\_tile/ liegen! (hardcoded)
                        \end{itemize}

\end{frame}

\subsection{Datenbank}


\begin{frame}{Postgres/PostGIS}
Postgresql-DB mit PostGIS-Erweiterung
\begin{itemize}
  \item PostGIS noch selbst zu kompilieren!
  \item Empfehlung: hstore-Extension
  \item Version: 8.3 oder 9.x, nicht 8.4!
	\item Wichtig: Tunen der DB, siehe postgres-Hilfe
\end{itemize}

\end{frame}

\begin{frame}{Hardware}
Hardware:

\begin{itemize}
  \item DB entweder in RAM
  \item oder auf SSD
    \pause
  \item Alles andere Sinnlos...
\end{itemize}

Selbst für Mitteleuropa:
\begin{itemize}
          \item 6-Core Phenom 2.8GHz
          \item 16GB Ram
          \item SSD mit 500 MB/sec und 85000 IOPS
          \item tiles und Rest auf 3 anderen Platten
\end{itemize}
\end{frame}

\begin{frame}{Füllen der Datenbank}
OSM-Daten:
\begin{itemize}
        \item Runterladen des ganzen Planet oder Auszügen von \url{http://geofabrik.de}
	\item ev. ausschneiden eines Gebiets mit \href{http://wiki.openstreetmap.org/wiki/Osmosis}{Osmosis}
	\item einspielen der Daten mit osm2pgsql
\end{itemize}
Höhenschichtlinien für Contour-Maps:
\begin{itemize}
        \item Daten für AT von der NASA downloaden
	\item Anleitung zum Importieren mittels scripts: \url{http://wiki.openstreetmap.org/wiki/Contours\_on\_the\_Cycle\_Map}
\end{itemize}

\end{frame}

\begin{frame}{Minutely Updates}
\begin{itemize}
        \item Es gibt minutely diffs des Planeten (.osc) am OSM-Server
	\item cron-job downloaded sie
 	\item diff-applien mit osm2pgsql
	\item bounding box bei osm2pgsql angeben.
\end{itemize}

\end{frame}


\section{Help?}

\begin{frame}{Help}

\begin{itemize}
  \item Erste Station sollte das Wiki sein: \url{http://wiki.osm.org}
  \item Immer noch etwas Unklar? $\Rightarrow$ Mailingliste \href{http://lists.openstreetmap.org/listinfo/talk-at}{talk-at}
  \pause
  \item \href{http://wiki.openstreetmap.org/wiki/Graz/Stammtisch}{Stammtisch}! In Graz alle 2 Monate - der nächste am 31. Mai im Brot \& Spiele
\end{itemize}

\end{frame}

\section{Danke!}

\begin{frame}{Abschluss}

Folien zum Mapnik-Vortrag für das \href{http://www.barcamp.at/BarCamp\_Graz\_2012}{GeoCamp/BarCamp Graz 2012} am 11.-13.5.2012.
\vspace{1cm}

Folien unter \includegraphics[width=1cm]{cc-by-sa.png}.
\vspace{1cm}

Erstellt mittels \LaTeX Beamer, source auf Anfrage.
\vspace{1cm}

\href{mailto:michael.maier@student.tugraz.at}{Michael Maier}
\end{frame}

\end{document}


\end{document}
